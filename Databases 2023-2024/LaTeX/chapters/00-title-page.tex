\thispagestyle{empty}
\begin{titlepage} 
    \thispagestyle{empty}
    \center
    \large{Δημοκρίτειο Πανεπιστήμιο Θράκης} \\ [0.2cm]
    \large {Τμήμα Ηλεκτρολόγων Μηχανικών και Μηχανικών Υπολογιστών} \\ [1.5cm]

    \begin{tabular}[t]{@{}l} 
      Ομάδα: 8
    \end{tabular}
    \hfill% move it to the right
    \begin{tabular}[t]{l@{}}
       Ιανουάριος 10, 2024
    \end{tabular}

    \vspace{2.0cm}
    \huge {\bfseries {Βάσεις Δεδομένων - Εργασία}} \\[0.2cm]
    \Large {\bfseries {\selectlanguage{english}{Database Driven Website - Moving On}}} \\[0.5cm]

    \selectlanguage{greek}
    \vspace{0.6cm}
    \begin{tabular}{ll}
        \large{Βασιλική Ειρήνη Μηλιούδη Σίσκου}    &  \large{ΧΧΧΧΧ} \\
        \large{Θανάσης Τσιρίκας}  &  \large{ΧΧΧΧΧ} \\
        \large{Παναγιώτα Νεφέλη Κουλιούμπα}     &  \large{ΧΧΧΧΧ} \\
    \end{tabular}

    \vspace{1.0cm}
    \begin{abstract}
        \noindent O σχεδιασμός της βάσης δεδομένων εστιάζει στην αποθήκευση πληροφοριών για μια μικρή μεταφορική εταιρεία. Μετά από ανάλυση απαιτήσεων, δημιουργήθηκαν τα διαγράμματα οντοτήτων–συσχετίσεων(Ο-Σ) και το σχεσιακό, τα οποία αποτέλεσαν τη βάση για την ανάπτυξη του κώδικα. Η εφαρμογή των παραπάνω θα έχει ως αποτέλεσμα τη δημιουργία ιστοσελίδας, η οποία θα ενσωματώνει μέρος της παραπάνω βάσης δεδομένων.
    \end{abstract}
\end{titlepage}