\noindentΕφαρμόζουμε το θεώρημα που λέει ότι η κλειστότητα του κλειστού καθορίζει όλα τα γνωρίσματα. Τα \foreignlanguage{english}{B, C, D, F, G, H, I} και \foreignlanguage{english}{J} απορρίπτονται γιατί εξαρτώνται από κάποιο άλλο γνώρισμα και επομένως αντί για αυτά μπορούμε να χρησιμοποιήσουμε το γνώρισμα από το οποίο εξαρτώνται.
Τελικά προκύπτει ότι το κλειδί της R θα είναι το: \foreignlanguage{english}{$\{A, E\}$}$^+$.
\hfill \break
2. Να σπάσετε την \foreignlanguage{english}{R} σε σχέσεις που να βρίσκονται σε \foreignlanguage{english}{2NF}